\documentclass{book}
\usepackage{multicol}
\usepackage{multirow}
\usepackage{geometry}
\usepackage{amssymb}


\title{Gechman, Marvin - Project management of large software-intensive systems _ controlling the software development process-CRC Press (2019)}
\author{Abouhamze Fahime}
		
\begin{document}
\chapter{}
\begin{multicols}{2}
product or system. Members of the SwIPT should provide
the following:
\begin{itemize}
	\renewcommand{\labelitemi}{\scriptsize$\blacksquare$}
	\item
	Share a common understanding of the SwIPTs tasks,
	objectives and responsibilities.
	\item
	Provide the technical skills and expertise needed to
	accomplish the tasks and objectives.
	\item
	Collaborate internally and externally with other
	SwIPTs and relevant stakeholders.
	\item
	Provide the advocacy and representation to address all
	phases of the life cycle.
\end{itemize}

	This Guidebook assumes SwIPTs are the basic performing
organizatin and that they have the resources and authority
tproduce and deliver software products and execute pro-
ceses. On very small projects the SwIPT may consist of just
one person who is assigned all the software tasks; on very
lrge systems the SwIPT could consist of hundreds of team
membrs as discussed in Subsection 1.6.2. The software proj-
eyou are managing may be composed of single SwIPT or
muliple SwIPTs.

	he Integrated Master Plan, as described in Section 8.1,
idenifies responsibilities for the SwIPTs. The project’s con-
trat should identify key personnel and their responsibilities
during software planning, development, test and transition
tdeployment. he IMP hould esignate software respon-
sibiiies through identification of program events, planned
acmplishmntas well as acceptance criteria for each
miletone.

	The organization of SwIPTs is normally consistent with
the product hierarchy as efined in he Contract Work
Breakon Structure. Primary esponsibility for each WBS ele-
ment should be assigned to a single SwIPT. This allows SwIPTs
tidentify clear and measurable outputs and interfaces; it
alo facilitates he flow down f requirements to he SwIPTs.
Olarge rojects, the wIPT would include a CSwE, proba-
bly a Chief Software Architect, plus a large team of Software
Devlpers, Tet Eginers anSbject Matter Experts.

	At the program level, there may be an over-arching
SwIPT led by he CSwE who eports directly o he Program
Manager. If a program does not ave CSwE, the func-
tions are performed by the person(s) having those responsi-
bilities regardless of their job title. If you have a government
contract, the Government Acquisition Team may have an
oversight organizational structure similar to yours, and the
SwIPT tasks may involve articipation y he overnment’s
SwIPT members.\\

\setcounter{section}{6}
\setcounter{subsection}{5}
\subsection{Interpersonal Communication}
No methodology or Software Process Improvement strategy can
overcome serious problems involving ineffective communication
or the mismanagement of interpersonal conflicts.

	Frequent and honest communication with your staff is a
critical factor in increasing the likelihood of project success
and the mitigation of problems. As the SPM, you should seek
cusomer involvement and encourage end-user input in the
developmet process to avoid misinterpretation of require-
ments, misunderstanding of changing customer needs,
and unrealistic expectations. Project Managers, Software
Dveloper, end-users, customers and project sponsors need
to communicate frequently.

	Bad news resulting from these conversations may be
to your advantage if the problems are communicated early
eogh for you to take Corrective Actions. Also, casual
covesatins with the customer, team members and other
stkehors may surface a potential problem sooner than its
apperance at formal meetings when the problem may be
much more severe.

	You have the right and obligation to criticize the work
underway as long as it is provided in a way that is construc-
tive, respectful and non-accusatory. Effective interpersonal
commictin, in an intellectually honest fashion, along
with effective team conflict management and resolution, are
the keys to successful Software Project Management. These
isses are discussed in more depth in Section 7.5.

\section{Customer Focus}
Without the customer, you don’t have a project!
The “customer” as referenced in this Guidebook refers to
thorganization responsible or the rogram’s contract includ-
ing funding and approval of the final products. It implicitly
iclues their representatives—such as consultants hired by
the customer. Also, you may interface daily with the custom-
es Project Management Office also called the “Acquisition
Program Office.”

	Even if the software-intensive project being developed is
for internal use by another division or group in your com-
pany, they are still your “customer” and are very likely fund-
ig the project through inter-departmental ransfer) nd
they deserve exactly the same treatment as an external cus-
tomer. The importance of customer focus is also discussed
iSubsection 7.7.2. As the Project Manager, it is self-evident
that you must figure out how to please your customer—even if
tey are wrong. The following short story is a personal exam-
ple to make that point.

Lessons Learned. Many years ago, when auto-
mation was at its infancy, was the Project
Manager of a pioneering effort to automate
library processes. We had a well-funded project
with a large government technical library, and I
needed to inerface ithe somer’s library

director, who was more than a bit autocratic and
knew nothing about software development.
At the kickoff meeting, I described the devel-
opment plan which included requirements analy-
sis and production of the Software Requirements
Specification (SRS). The library director imme-
diately objected and said there is no need for
requirements analysis, that it is a complete waste
of time, as all we needed to do was to “automate
the current processes.”
I did not object since it is not a good idea to
argue with your customer (especially this one).
So, this is what we did. We identified the rela-
tively simple tasks where what needed to be done
was clearly needed, and I assigned Programmers
to produce the design and code for those tasks.
In subsequent status reports that was the output
shown to the library director so she would see
physical progress.
Meanwhile, quietly in the “back room,” a
group of us worked on the system and software
requirements resulting in an abridged version of
the SRS. his was necessary because e eeded
to understand exactly what we were building
since no one had prior experience with library
automaton at thtime.
The bottom line is that if you have a tech-
nically unsophisticated customer, who impedes
your ability to follow the required processes,
use your ingenuity to figure out ways to work-
around these obstacles to get the job done. We
did that in this example, and t was a very suc-
cessful pioneering library automation imple-
mentation, delivered on time and within the
budget.

\section{Software Classes and Categories}
Project managers must make sure there is a mechanism in
pacto assign the proper designation to each software entity.
There are typically three generic classes of software in a soft-
ware-intensive system:
\begin{itemize}
	\renewcommand{\labelitemi}{\scriptsize$\blacksquare$}
	\item System Critical Software (SCS) described in Subsection
	1.8.1
	\item Support Software (SS) described in Subsection 1.8.2
	\item Commercial Off-the-Shelf or Reuse software (C/R)
	described in Subsection 1.8.3
\end{itemize}
Each software generic class can be further subdivided into
categories as needed for your program, resulting in the identi-
fication of 4–8 categories of software for a typical program.
The number of software classes, the number of categories
within those classes, and the names of each are not critical.
What is important is that there must be definition of the
category assigned to each software entity because not every soft-
ware ntity eeds o have he full set of documentation, the full
set of reviews, the full set of metrics, and the same level of testing.
Assigning categories to software entities, such as the exam-
ple categories shown in the following Tables 1.8–1.10, can
result in major time and cost savings by eliminating unnecessary
documents, reviews, metrics nd testing. However, the simplicity
of this approach is deceiving since obtaining agreements from
all stakeholders on the appropriate category to assign each soft-
ware entity is not always simple or obvious. As the Software
Project Manager, you must work it out and obtain concurrence.

\subsection{System Critical Software}
System Critical Software, often called Mission Critical soft-
ware, is physically part of, dedicated to, and/or essential to full
performance of the system. SCS can be defined as:
System Critical Software consists of software func-
tions that f not erformed, performed out-of-
sequence, or performed incorrectly, may directly or
indirectly cause the system to fail.
System Critical Software plays a critical role in the over-
all dependability, reliability, maintainability and availabil-
ity of every system. SCS ay be xpanded to two software
\end{multicols}

\setcounter{table}{7}
\begin{table}
	\caption{System Critical Software Class and Category Definitions}
	\begin{tabular}[m]{|l|l|l|}
		\hline
		\textbf{Class Definition} & \textbf{Category} & \textbf{Category Definition} \\ 
		\multirow{2}{*}{SYSTEM CRITICAL SOFTWARE All applications software used to perform real-time operations and non-real-time functions implicitly required to implement the system functions allocated to software.} 
		& SCS-1 
		& Deliverable applications software that has a direct role in system operation required for full system functionality. \\
		& SCS-2 
		& Firmware is software embedded in the deliverable hardware. Firmware is treated in the same way as software that executes in general purpose computers. \\
	\end{tabular}
\end{table}

\begin{table}
	\caption{Support Software Class and Category Definitions}
	\begin{tabular}[m]{|l|l|l|}
		Class Definition & Category & Category Definition\\
		\multirow{3}{*}{SS
			SUPPORT SOFTWARE
			Software that aids in system hardware and
			software development, integration,
			qualification, operations, test and maintenance}
		& SS-1
		& Software Items that play a direct role in program and
		system development including software and system
		requirements qualification and acceptance testing for
		final “sell-off.” \\
		& SS-2
		& Support software that is typically prototype software,
		simulation software, or performance analysis and
		modeling tools (although some of this type of
		software may be selected to be in Category SS-1).\\
		& SS-3
		& Non-deliverable and non-critical tools or test drivers
		that indirectly aid in the development of the other
		categories of software. \\
	\end{tabular}
\end{table}

\begin{table}
	\caption{COTS/Reuse Software Class and Category Definitions}
	\begin{tabular}[m]{|l|l|l|}
		Class Definition & Category & Category Definition \\
		\multirow{2}{*}{C/R
			COTS/REUSE SOFTWARE
			Non-developmental Software Items including Commercial
			Off-the-Shelf software.
			All C/R products must be treated and controlled as defined
			for the category targeted for its end use.}
		& C/R-1
		Non-developmental software that is
		unmodified COTS or Reused software. \\
		& C/R-2
		& Non-developmental software that is
		modified COTS or Reused software.(*) (A
		distinction between vendor-provided and
		internally reused software can be made if
		meaningful to your program). \\
	\end{tabular}
\end{table}

(*) Modifying vendor-provided COTS is a high-risk approach and is not recommended; however, modifying internally
reused software can be effective since you have access to the full source code.

\begin{multicols}{2}
	
categories to specifically identify firmware as shown by the
example in Table 1.8. If it is meaningful to your program,
the number of SCS categories can be further expanded. The
SCS implementation rocess is described later in Subsection
3.5.1.

\subsection{Support Software}
Support Software (SS) aids in system hardware and software
development, test, integration, qualification and mainte-
nance. The SS class may be composed of three Software Item
(SI) categories, or example, SS-1, S-2 and SS-3 as defined
in Table 1.9.
SCS-1, SCS-2 and SS-1 software categories (but not SS-2
or SS-3) are usually deliverable and contractually obligated.
These three ategories ust pass through all f the develop-
mental phases, including all of the relevant software docu-
mentation, reviews, metrics, and testing, and are subject to
external Software Discrepancy Reports (SDRs), also called
problem reports, see Subsection .4.1).
SS-2 software is used in non-operational environments
and is normally not deliverable. Both SS-2 and SS-3 software
categories do not go through the full software life cycle or
receive external SDRs.

In some cases, important Support Software packages
may be contractually deliverable. For example, deliverable
Support Software may include training software, database-
related software, software used in automatic test equipment,
and simulation software used for diagnostic purposes dur-
ing the sustainment activity. The Developers must decide the
appropriate category for all software entities in compliance
with contractual requirements. The Software Support process
is described in Subsection 3.8.2.

\subsection{Commercial Off-the-Shelf}
and eused Software
COTS/Reused (C/R) software is non-developmental
Software Items and is often referred to as third-party soft-
ware. It includes Commercial and Government off-the-
shelf (COTS or GOTS) software as well as reused software
obtained from endors r nternal libraries, previously devel-
oped, or developed y other programs, set up pecifically or
reuse. “COTS” is a generic term and does not realistically
mean you can go to a store and buy it off-the-shelf. The C/R
class may be composed of two categories as described by the
example n able 1.10 r dditional categories if eaningful
to your project.

\subsection{Software Category Features}
A single SI may consist of different classes and/or categories.
In that event, each part of the SI must be compliant with the
documentation, review and testing requirements of the cate-
gory assigned to it. All software releases must be configuration
controlled in a Software Development Library (SDL) at the
subsystem level or by the Master Software Development Library
(MSDL) at the program level as described in Section 9.3.
If we view the categories as hierarchical levels, SCS is
the top level, SS is the middle level and C/R the lowest level.
Software cannot be moved up or “promoted” to a higher cat-
egory level without additional development and testing. To
achieve a higher category level, the software must be “re-engi-
neered” and conform to the documentation, review and testing
requirements imposed on the igher category level. All COTS
and reused products must be treated and controlled as defined
for the category targeted for its end se. COTS software is also
discussed with Software Sustainment in Chapter 16.

\subsection{Software Development}
Standards and Regulations
Applicable software standards and practices must be
addressed in the SDP, or other documents such as a oftware
Standards and Practices Manual. That documentation should
identify the programming language standards to be used
including specific ersions, a ist of pplicable standards dic-
tated by the customer, toolsets to be used, operational details
of the program’s defined software process, use of COTS/
Reuse software, the process for waivers or deviations, and
portions of the applicable standards that need to be modified
for your project.
These standards ensure that Developers produce con-
sistent software development products. Standards also help
ensure he similarity of the structure of all code/design units
so that lines of code counts and software measurements can
be applied consistently. Standards must provide value added
to your project.
Software tandards are discussed or referred to in many
sections of this Guidebook, they are addressed in Section
4.5, and some common standards are listed in Appendix I.
Standards are adjustable to your needs, but regulations are
not; hey must be followed. egulations re imposed by
government bodies, and if you don’t comply with them you
could be fined or even face imprisonment. For example:

\begin{itemize}
	\renewcommand{\labelitemi}{\scriptsize$\blacksquare$}
	\item Health Insurance Portability and Accountability Act
	(HIPPA).
	\item Occupational Safety and Health Administration
	(OSHA) requires all organizations to provide for the
	protection and safety of all of your employees.
	\item Sarbanes-Oxley Compliance (SOX) Act of 2002 affects
	all publicly held corporations.
\end{itemize}

In addition to government regulations, there are all kinds
of industry-specific regulations and laws that your project,
and your company, must adhere to. As the Software Project
Manager, you need to be aware of the mandatory regula-
tions, and those that can or may affect your project, and
incorporate provisions for these concerns into your SDP and
other planning documents.

\section{Why Software-Intensive}
Systems Fail and Overrun
The Standish Group has been periodically publishing their
well-known CHAOS eport Standish Group, 2016) con-
taining software project success/failure/challenged rates start-
ing in 1994. They have lso ttempted o nalyze success
factors at each publication. The rating of the top success fac-
tors change ith each eport. owever the perennial major
factors needed for project success appear to be:

\begin{itemize}
	\renewcommand{\labelitemi}{\scriptsize$\blacksquare$}
	\item Executive management support (sponsorship)
	\item Customer (user) involvement
	\item Clear Objectives and requirements
	\item Optimizing scope with horter roject milestone dura-
	tions (part of planning)
	\item Skilled staff
	\item Project Manager expertise
\end{itemize}

This means the absence of those factors are reasons for
failure. The past few CHAOS reports have found a decrease
in IT project uccess rates nd n increase in IT project fail-
ue rates where:
\begin{itemize}
	\renewcommand{\labelitemi}{\scriptsize$\blacksquare$}
	\item 32\% to 35\% were considered successful projects as they
	were ompleted on time, on udget nd with the
	required features and functions (the Standish defini-
	tion of success).
	\item 19\% to 24\% f the projects were considered failures
	because hey were cancelled before they were om-
	pleted, or they were delivered but never used.
	\item 44\% to 46\% were considered challenged meaning they
	were either finished late, over budget, or with fewer
	than the equired eatures nd functions.
\end{itemize}

If the data is accurate, based on their definition of suc-
cess, 65\% to 68\% of the IT oftware rojects were not suc
cessful. Software tools provided by the project management
industry is estimated to be over \$3 billion annually and pre-
dicted to boom to over \$5 billion in a few years, but these

new methodologies and software management packages
don’t seem to have any impact. Maybe the real solution is to
create better Software Project Managers (achievable if they
read this Guidebook!). Some serious researchers point out
problems with the CHAOS reports, most notably:
\begin{itemize}
	\renewcommand{\labelitemi}{\scriptsize$\blacksquare$}
	\item Unlike published academic research, the data needed
	to evaluate the claims is kept private so their data or
	methods cannot be independently verified.
	\item Their definition of success is narrow; success means
	cost, time and content were accurately estimated up
	front; if the project did not meet those estimates, it was
	categorized as challenged. In the real world, successful
	projects usually undergo changes as they are developed,
	and they may take longer to incorporate new function-
	ality not initially planned for, and the resulting project
	may indeed be successful.
\end{itemize}

Popular, and often quoted, reports that contain question-
able data is not new phenomenon as described by a personal
experience below.

\textbf{Lessons Learned.} The cost department at a
company I orked for decided o find out what
it cost to develop software. They sent out a ques-
tionnaire to a large number of completed soft-
ware-intensive programs, and one key question
asked or he ost of the oftware portion of their
program. Their final report was very professional
and thorough, and it was referenced widely in
the industry.
The big mistake made by the cost depart-
ment was the Cost Analysis Team that prepared
the report did not have any Software Engineers
either on the team or used as a consultant. As a
result, the key question about software cost did
not identify the scope of what to include in the
cost figures. ome programs surveyed considered
all the equirements analysis and integration
tasks to be part of Systems Engineering and did
not count it as a software cost. Some programs
charge Software Quality Assurance (SQA) and
Software Configuration Management (SCM) to
those departments. Other programs onsider all
of those tasks part of the software effort. Since the
data collected had no consistency in scope, the
well documented cost results were meaningless.

\begin{table}
	\caption{Core Success Factors for Software-Intensive
		System Implementations}
	\begin{enumerate}
		\item Obtain Senior Management Commitment to Your Project
		\item  Define and Document Your Customer’s System and Software Requirements
		\item  Prepare and Follow a Software Development Plan and a Defined Process
		\item Prepare and Follow a Work Breakdown Structure
		\item Hire the Best Programmers and Support Staff Available and Provide Training
		\item Select and Document a Clear Software Development Methodology
		\item Take a Big Picture Systems Approach
		\item Measure and Track Quality, Performance and Progress
		\item Build Quality Into Software from the Start of the Life Cycle
		\item Identify, Manage and Mitigate Software Risks
		\item If You Change Anything When Developing a Complex System: Retest
		\item Prepare Software Documentation as Needed by Your Project
		\item Provide Configuration Management of the Work Products
		\item Don’t Make Decisions Based Only on Budget or Schedule Issues
		\item Collaborate With Hardware and System Engineers
		\item Use Software Standards Where Applicable
		\item Document Lessons Learned to Facilitate Process Improvement
		\item Lead, Motivate and Manage Effectively and Ethically
		\item Keep Solutions Simple and Concise and Avoid Gold Plating
		\item Do Not Concur With Unrealistic Schedules or Budgets
		\item Don’t Take Risky Shortcuts
		\item Always Use Common Sense
	\end{enumerate}
\end{table}

Core Success Factors. Why do software-intensive projects
fail, or result in severe cost overruns, or are delivered well
beyond the planned completion date? The principal answer
is: they do not follow all of the core success factors needed for
successful software-intensive system implementations. My
list of the core uccess factors in able 1.11 re not listed
in order of importance. They are all important. I am sure
you can add more to the list, but these are the ssential ele-
ments for successful development and deployment of large
software-intensive systems. Memorize this list of 22 success
factors or ape it to he wall near our desk.
 
\end{multicols}
\end{document}		
		
	    
		
		
